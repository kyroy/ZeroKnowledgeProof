\documentclass[a4paper,parskip=half]{scrartcl}

\usepackage[english]{babel}
\usepackage[T1]{fontenc}
\usepackage[utf8]{inputenc}

\usepackage{graphicx}
\usepackage{amsmath}

\usepackage[usenames,dvipsnames]{color}

\usepackage[
	colorinlistoftodos
	, textsize=small
	%, disable
]{todonotes}

\usepackage{hyperref}
\usepackage{calc}
\hypersetup{
    colorlinks,
    citecolor=black,
    filecolor=black,
    linkcolor=blue,
    urlcolor=blue%black
}

\title{Zero-Knowledge Proofs on the Ethereum blockchain}
\subtitle{Exposé for Master Thesis}
\author{Dennis Kuhnert}
\date{\today}
\publishers{\begin{minipage}[t]{0.25\textwidth}%
		\flushright\Large%
		Supervisors:~%
	\end{minipage}%
	\vspace{0pt}~%
	\begin{minipage}[t]{0.5\textwidth}%
		\flushleft\Large%
		Prof. Dr.-Ing. Stefan Tai\\%
		M.Sc. Jacob Eberhard
	\end{minipage}}

\begin{document}
\maketitle


\section{Zero-Knowledge Proofs}

Usually, proving the knowledge of a solution for a problem requires the disclosure of the solution itself, or at least parts of it.
Goldwasser, Micali and Rackoff define zero-knowledge proofs ``as those proofs that convey no additional knowledge other than the correctness of the proposition in question''~\cite{goldwasser1989knowledge}, here the knowledge of a solution.
Most of the existing zero-knowledge proofs are interactive, meaning they are protocols between two parties, where the \emph{prover} tries to convince the \emph{verifier}.

\begin{itemize}
\item first proposed - Goldwasser, Micali and Rackoff \cite{goldwasser1989knowledge}
\item motivation
	\begin{itemize}
	\item public data
	\end{itemize}
\item related work
	\begin{itemize}
	\item vitalik buterin \cite{buterin2015public}
	\item interesting topic, ppl try to develop new software (--addons-- to ethereum)
	\end{itemize}
\item plan
	\begin{itemize}
	\item reference implementations, like 3 coloring problem
	\item implement Feige-Fiat-Shamir -> see if secure - Raffo and Kizza \cite{raffo2002digital, kizza2010feige}
	\item NOT to solve/develop new zero-knowledge proofs
	\item interactive zkp
	\end{itemize}
\end{itemize}

\section{Related Work}

\bibliographystyle{alpha}
\bibliography{../references}
\end{document}
