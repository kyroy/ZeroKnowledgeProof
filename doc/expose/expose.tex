\documentclass[a4paper]{scrartcl}

\usepackage[english]{babel}
\usepackage[T1]{fontenc}
\usepackage[utf8]{inputenc}

\usepackage{graphicx}
\usepackage{amsmath}

\usepackage[usenames,dvipsnames]{color}

\usepackage[
	colorinlistoftodos
	, textsize=small
	%, disable
]{todonotes}

\usepackage{hyperref}
\usepackage{calc}
\hypersetup{
    colorlinks,
    citecolor=Green,
    filecolor=black,
    linkcolor=blue,
    urlcolor=blue%black
}

\title{Zero-Knowledge Proofs}
\subtitle{Exposé for Master Thesis}
\author{Dennis Kuhnert}
\date{\today}
\publishers{\begin{minipage}[t]{0.25\textwidth}%
		\flushright\Large%
		Supervisors:~%
	\end{minipage}%
	\vspace{0pt}~%
	\begin{minipage}[t]{0.5\textwidth}%
		\flushleft\Large%
		Prof. Dr.-Ing. Stefan Tai\\%
		M.Sc. Jacob Eberhard
	\end{minipage}}

\begin{document}
\maketitle

\section{Standard Forms}
Typically, proofs of properties of Distributed Consensus algorithms argue about the global state of the whole system.
But practically, these algorithms are implemented in action-based formalisms like process calculi.
These can be used to program the behaviour of a local process in a comfortable and intuitive way, as they are closer to the common programming languages.
Thus arguing for global properties of a distributed algorithm can be very cumbersome.
\cite{bengtson2009psi}

\bibliographystyle{alpha}
\bibliography{../references}
\end{document}
